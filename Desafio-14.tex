% Options for packages loaded elsewhere
\PassOptionsToPackage{unicode}{hyperref}
\PassOptionsToPackage{hyphens}{url}
%
\documentclass[
]{article}
\usepackage{amsmath,amssymb}
\usepackage{iftex}
\ifPDFTeX
  \usepackage[T1]{fontenc}
  \usepackage[utf8]{inputenc}
  \usepackage{textcomp} % provide euro and other symbols
\else % if luatex or xetex
  \usepackage{unicode-math} % this also loads fontspec
  \defaultfontfeatures{Scale=MatchLowercase}
  \defaultfontfeatures[\rmfamily]{Ligatures=TeX,Scale=1}
\fi
\usepackage{lmodern}
\ifPDFTeX\else
  % xetex/luatex font selection
\fi
% Use upquote if available, for straight quotes in verbatim environments
\IfFileExists{upquote.sty}{\usepackage{upquote}}{}
\IfFileExists{microtype.sty}{% use microtype if available
  \usepackage[]{microtype}
  \UseMicrotypeSet[protrusion]{basicmath} % disable protrusion for tt fonts
}{}
\makeatletter
\@ifundefined{KOMAClassName}{% if non-KOMA class
  \IfFileExists{parskip.sty}{%
    \usepackage{parskip}
  }{% else
    \setlength{\parindent}{0pt}
    \setlength{\parskip}{6pt plus 2pt minus 1pt}}
}{% if KOMA class
  \KOMAoptions{parskip=half}}
\makeatother
\usepackage{xcolor}
\usepackage[margin=1in]{geometry}
\usepackage{color}
\usepackage{fancyvrb}
\newcommand{\VerbBar}{|}
\newcommand{\VERB}{\Verb[commandchars=\\\{\}]}
\DefineVerbatimEnvironment{Highlighting}{Verbatim}{commandchars=\\\{\}}
% Add ',fontsize=\small' for more characters per line
\usepackage{framed}
\definecolor{shadecolor}{RGB}{248,248,248}
\newenvironment{Shaded}{\begin{snugshade}}{\end{snugshade}}
\newcommand{\AlertTok}[1]{\textcolor[rgb]{0.94,0.16,0.16}{#1}}
\newcommand{\AnnotationTok}[1]{\textcolor[rgb]{0.56,0.35,0.01}{\textbf{\textit{#1}}}}
\newcommand{\AttributeTok}[1]{\textcolor[rgb]{0.13,0.29,0.53}{#1}}
\newcommand{\BaseNTok}[1]{\textcolor[rgb]{0.00,0.00,0.81}{#1}}
\newcommand{\BuiltInTok}[1]{#1}
\newcommand{\CharTok}[1]{\textcolor[rgb]{0.31,0.60,0.02}{#1}}
\newcommand{\CommentTok}[1]{\textcolor[rgb]{0.56,0.35,0.01}{\textit{#1}}}
\newcommand{\CommentVarTok}[1]{\textcolor[rgb]{0.56,0.35,0.01}{\textbf{\textit{#1}}}}
\newcommand{\ConstantTok}[1]{\textcolor[rgb]{0.56,0.35,0.01}{#1}}
\newcommand{\ControlFlowTok}[1]{\textcolor[rgb]{0.13,0.29,0.53}{\textbf{#1}}}
\newcommand{\DataTypeTok}[1]{\textcolor[rgb]{0.13,0.29,0.53}{#1}}
\newcommand{\DecValTok}[1]{\textcolor[rgb]{0.00,0.00,0.81}{#1}}
\newcommand{\DocumentationTok}[1]{\textcolor[rgb]{0.56,0.35,0.01}{\textbf{\textit{#1}}}}
\newcommand{\ErrorTok}[1]{\textcolor[rgb]{0.64,0.00,0.00}{\textbf{#1}}}
\newcommand{\ExtensionTok}[1]{#1}
\newcommand{\FloatTok}[1]{\textcolor[rgb]{0.00,0.00,0.81}{#1}}
\newcommand{\FunctionTok}[1]{\textcolor[rgb]{0.13,0.29,0.53}{\textbf{#1}}}
\newcommand{\ImportTok}[1]{#1}
\newcommand{\InformationTok}[1]{\textcolor[rgb]{0.56,0.35,0.01}{\textbf{\textit{#1}}}}
\newcommand{\KeywordTok}[1]{\textcolor[rgb]{0.13,0.29,0.53}{\textbf{#1}}}
\newcommand{\NormalTok}[1]{#1}
\newcommand{\OperatorTok}[1]{\textcolor[rgb]{0.81,0.36,0.00}{\textbf{#1}}}
\newcommand{\OtherTok}[1]{\textcolor[rgb]{0.56,0.35,0.01}{#1}}
\newcommand{\PreprocessorTok}[1]{\textcolor[rgb]{0.56,0.35,0.01}{\textit{#1}}}
\newcommand{\RegionMarkerTok}[1]{#1}
\newcommand{\SpecialCharTok}[1]{\textcolor[rgb]{0.81,0.36,0.00}{\textbf{#1}}}
\newcommand{\SpecialStringTok}[1]{\textcolor[rgb]{0.31,0.60,0.02}{#1}}
\newcommand{\StringTok}[1]{\textcolor[rgb]{0.31,0.60,0.02}{#1}}
\newcommand{\VariableTok}[1]{\textcolor[rgb]{0.00,0.00,0.00}{#1}}
\newcommand{\VerbatimStringTok}[1]{\textcolor[rgb]{0.31,0.60,0.02}{#1}}
\newcommand{\WarningTok}[1]{\textcolor[rgb]{0.56,0.35,0.01}{\textbf{\textit{#1}}}}
\usepackage{graphicx}
\makeatletter
\newsavebox\pandoc@box
\newcommand*\pandocbounded[1]{% scales image to fit in text height/width
  \sbox\pandoc@box{#1}%
  \Gscale@div\@tempa{\textheight}{\dimexpr\ht\pandoc@box+\dp\pandoc@box\relax}%
  \Gscale@div\@tempb{\linewidth}{\wd\pandoc@box}%
  \ifdim\@tempb\p@<\@tempa\p@\let\@tempa\@tempb\fi% select the smaller of both
  \ifdim\@tempa\p@<\p@\scalebox{\@tempa}{\usebox\pandoc@box}%
  \else\usebox{\pandoc@box}%
  \fi%
}
% Set default figure placement to htbp
\def\fps@figure{htbp}
\makeatother
\setlength{\emergencystretch}{3em} % prevent overfull lines
\providecommand{\tightlist}{%
  \setlength{\itemsep}{0pt}\setlength{\parskip}{0pt}}
\setcounter{secnumdepth}{-\maxdimen} % remove section numbering
\usepackage{bookmark}
\IfFileExists{xurl.sty}{\usepackage{xurl}}{} % add URL line breaks if available
\urlstyle{same}
\hypersetup{
  pdftitle={Desafio 14 - ME315},
  pdfauthor={Bruce Trevisan},
  hidelinks,
  pdfcreator={LaTeX via pandoc}}

\title{Desafio 14 - ME315}
\author{Bruce Trevisan}
\date{2025-10-31}

\begin{document}
\maketitle

\section{1. Importar o conjunto de dados e
bibliotecas}\label{importar-o-conjunto-de-dados-e-bibliotecas}

\begin{Shaded}
\begin{Highlighting}[]
\ImportTok{import}\NormalTok{ pandas }\ImportTok{as}\NormalTok{ pd}
\ImportTok{from}\NormalTok{ plotnine }\ImportTok{import}\NormalTok{ ggplot, aes, geom\_point, geom\_smooth, labs, theme\_minimal, coord\_cartesian, facet\_wrap, theme\_light, theme, facet\_grid, theme\_bw, element\_text}
\ImportTok{from}\NormalTok{ plotnine.data }\ImportTok{import}\NormalTok{ diamonds}
\end{Highlighting}
\end{Shaded}

\begin{Shaded}
\begin{Highlighting}[]
\NormalTok{diamonds.head()}
\end{Highlighting}
\end{Shaded}

\begin{verbatim}
##    carat      cut color clarity  depth  table  price     x     y     z
## 0   0.23    Ideal     E     SI2   61.5   55.0    326  3.95  3.98  2.43
## 1   0.21  Premium     E     SI1   59.8   61.0    326  3.89  3.84  2.31
## 2   0.23     Good     E     VS1   56.9   65.0    327  4.05  4.07  2.31
## 3   0.29  Premium     I     VS2   62.4   58.0    334  4.20  4.23  2.63
## 4   0.31     Good     J     SI2   63.3   58.0    335  4.34  4.35  2.75
\end{verbatim}

\section{2. Associação entre peso (carat) e preço
(price)}\label{associauxe7uxe3o-entre-peso-carat-e-preuxe7o-price}

\begin{Shaded}
\begin{Highlighting}[]
\NormalTok{(ggplot(diamonds,aes(x}\OperatorTok{=}\StringTok{\textquotesingle{}carat\textquotesingle{}}\NormalTok{, y}\OperatorTok{=}\StringTok{\textquotesingle{}price\textquotesingle{}}\NormalTok{)) }\OperatorTok{+}
\NormalTok{geom\_point(alpha}\OperatorTok{=}\FloatTok{0.3}\NormalTok{, color}\OperatorTok{=}\StringTok{\textquotesingle{}steelblue\textquotesingle{}}\NormalTok{) }\OperatorTok{+}
\NormalTok{geom\_smooth(method}\OperatorTok{=}\StringTok{\textquotesingle{}lm\textquotesingle{}}\NormalTok{, color}\OperatorTok{=}\StringTok{\textquotesingle{}red\textquotesingle{}}\NormalTok{, se}\OperatorTok{=}\VariableTok{False}\NormalTok{) }\OperatorTok{+}
\NormalTok{labs(title}\OperatorTok{=}\StringTok{\textquotesingle{}Relação entre peso (carat) e preço (price)\textquotesingle{}}\NormalTok{,}
\NormalTok{     x}\OperatorTok{=}\StringTok{\textquotesingle{}Peso (carat)\textquotesingle{}}\NormalTok{,}
\NormalTok{     y}\OperatorTok{=}\StringTok{\textquotesingle{}Preço (USD)\textquotesingle{}}\NormalTok{) }\OperatorTok{+}
\NormalTok{coord\_cartesian(ylim}\OperatorTok{=}\NormalTok{(}\DecValTok{0}\NormalTok{, }\DecValTok{21000}\NormalTok{)) }\OperatorTok{+}
\NormalTok{theme\_minimal())}
\end{Highlighting}
\end{Shaded}

\pandocbounded{\includegraphics[keepaspectratio]{Desafio-14_files/figure-latex/unnamed-chunk-3-1.pdf}}

O gráfico revela uma relação positiva nítida entre peso (carat) e preço
do diamante. A dispersão do preço aumenta com o peso, evidenciando a
influência de outros atributos (corte, clareza).

\section{3. Relação entre preço e peso considerando corte
(cut)}\label{relauxe7uxe3o-entre-preuxe7o-e-peso-considerando-corte-cut}

\begin{Shaded}
\begin{Highlighting}[]
\NormalTok{(ggplot(diamonds, aes(x}\OperatorTok{=}\StringTok{\textquotesingle{}carat\textquotesingle{}}\NormalTok{, y}\OperatorTok{=}\StringTok{\textquotesingle{}price\textquotesingle{}}\NormalTok{, color}\OperatorTok{=}\StringTok{\textquotesingle{}cut\textquotesingle{}}\NormalTok{)) }\OperatorTok{+}
\NormalTok{ geom\_point(alpha}\OperatorTok{=}\FloatTok{0.3}\NormalTok{) }\OperatorTok{+}
\NormalTok{ geom\_smooth(method}\OperatorTok{=}\StringTok{\textquotesingle{}lm\textquotesingle{}}\NormalTok{, color}\OperatorTok{=}\StringTok{\textquotesingle{}black\textquotesingle{}}\NormalTok{, se}\OperatorTok{=}\VariableTok{False}\NormalTok{) }\OperatorTok{+}
\NormalTok{ facet\_wrap(}\StringTok{\textquotesingle{}\textasciitilde{}cut\textquotesingle{}}\NormalTok{) }\OperatorTok{+}
\NormalTok{ labs(title}\OperatorTok{=}\StringTok{\textquotesingle{}Relação entre preço e peso por tipo de corte\textquotesingle{}}\NormalTok{,}
\NormalTok{      x}\OperatorTok{=}\StringTok{\textquotesingle{}Peso (carat)\textquotesingle{}}\NormalTok{,}
\NormalTok{      y}\OperatorTok{=}\StringTok{\textquotesingle{}Preço (USD)\textquotesingle{}}\NormalTok{) }\OperatorTok{+}
\NormalTok{ coord\_cartesian(ylim}\OperatorTok{=}\NormalTok{(}\DecValTok{0}\NormalTok{, }\DecValTok{21000}\NormalTok{)) }\OperatorTok{+}
\NormalTok{ theme\_light() }\OperatorTok{+}
\NormalTok{ theme(legend\_position}\OperatorTok{=}\StringTok{\textquotesingle{}none\textquotesingle{}}\NormalTok{))}
\end{Highlighting}
\end{Shaded}

\pandocbounded{\includegraphics[keepaspectratio]{Desafio-14_files/figure-latex/unnamed-chunk-4-3.pdf}}

A relação entre preço e peso se mantém positiva em todos os tipos de
corte, mas o grau dessa relação varia. Diamantes com cortes de melhor
qualidade, como Ideal e Premium, tendem a apresentar preços mais altos
para o mesmo peso. Isso indica que o corte tem um papel significativo na
precificação, além do peso.

\section{4. Relação entre preço e peso considerando corte, cor e
clareza}\label{relauxe7uxe3o-entre-preuxe7o-e-peso-considerando-corte-cor-e-clareza}

\begin{Shaded}
\begin{Highlighting}[]
\NormalTok{(ggplot(diamonds, aes(x}\OperatorTok{=}\StringTok{\textquotesingle{}carat\textquotesingle{}}\NormalTok{, y}\OperatorTok{=}\StringTok{\textquotesingle{}price\textquotesingle{}}\NormalTok{, color}\OperatorTok{=}\StringTok{\textquotesingle{}color\textquotesingle{}}\NormalTok{)) }\OperatorTok{+}
\NormalTok{ geom\_point(alpha}\OperatorTok{=}\FloatTok{0.3}\NormalTok{) }\OperatorTok{+}
\NormalTok{ geom\_smooth(method}\OperatorTok{=}\StringTok{\textquotesingle{}lm\textquotesingle{}}\NormalTok{, se}\OperatorTok{=}\VariableTok{False}\NormalTok{) }\OperatorTok{+}
\NormalTok{ facet\_grid(}\StringTok{\textquotesingle{}cut \textasciitilde{} clarity\textquotesingle{}}\NormalTok{) }\OperatorTok{+}
\NormalTok{ labs(title}\OperatorTok{=}\StringTok{\textquotesingle{}Preço vs Peso por corte, cor e clareza\textquotesingle{}}\NormalTok{,}
\NormalTok{      x}\OperatorTok{=}\StringTok{\textquotesingle{}Peso (carat)\textquotesingle{}}\NormalTok{,}
\NormalTok{      y}\OperatorTok{=}\StringTok{\textquotesingle{}Preço (USD)\textquotesingle{}}\NormalTok{,}
\NormalTok{      color}\OperatorTok{=}\StringTok{\textquotesingle{}Cor\textquotesingle{}}\NormalTok{) }\OperatorTok{+}
\NormalTok{ coord\_cartesian(ylim}\OperatorTok{=}\NormalTok{(}\DecValTok{0}\NormalTok{, }\DecValTok{21000}\NormalTok{)) }\OperatorTok{+}
\NormalTok{ theme\_bw() }\OperatorTok{+}
\NormalTok{ theme(}
\NormalTok{     axis\_text\_x}\OperatorTok{=}\NormalTok{element\_text(rotation}\OperatorTok{=}\DecValTok{45}\NormalTok{, hjust}\OperatorTok{=}\DecValTok{1}\NormalTok{),}
\NormalTok{     figure\_size}\OperatorTok{=}\NormalTok{(}\DecValTok{14}\NormalTok{, }\DecValTok{8}\NormalTok{)}
\NormalTok{ ))}
\end{Highlighting}
\end{Shaded}

\begin{verbatim}
## C:\Users\trevi\AppData\Local\Programs\Python\PYTHON~1\Lib\site-packages\plotnine\stats\stat_smooth.py:213: PlotnineWarning: Smoothing requires 2 or more points. Got 1. Not enough points for smoothing. If this message a surprise, make sure the column mapped to the x aesthetic has the right dtype.
## C:\Users\trevi\AppData\Local\Programs\Python\PYTHON~1\Lib\site-packages\plotnine\stats\stat_smooth.py:213: PlotnineWarning: Smoothing requires 2 or more points. Got 1. Not enough points for smoothing. If this message a surprise, make sure the column mapped to the x aesthetic has the right dtype.
## C:\Users\trevi\AppData\Local\Programs\Python\PYTHON~1\Lib\site-packages\plotnine\stats\stat_smooth.py:213: PlotnineWarning: Smoothing requires 2 or more points. Got 1. Not enough points for smoothing. If this message a surprise, make sure the column mapped to the x aesthetic has the right dtype.
## C:\Users\trevi\AppData\Local\Programs\Python\PYTHON~1\Lib\site-packages\plotnine\stats\stat_smooth.py:213: PlotnineWarning: Smoothing requires 2 or more points. Got 1. Not enough points for smoothing. If this message a surprise, make sure the column mapped to the x aesthetic has the right dtype.
## C:\Users\trevi\AppData\Local\Programs\Python\PYTHON~1\Lib\site-packages\plotnine\stats\stat_smooth.py:213: PlotnineWarning: Smoothing requires 2 or more points. Got 1. Not enough points for smoothing. If this message a surprise, make sure the column mapped to the x aesthetic has the right dtype.
\end{verbatim}

\pandocbounded{\includegraphics[keepaspectratio]{Desafio-14_files/figure-latex/unnamed-chunk-5-5.pdf}}

Ao incluir as variáveis de corte, cor e clareza, percebe-se que a
relação entre peso e preço continua positiva, mas apresenta diferentes
padrões de inclinação e dispersão. Diamantes de cor mais pura (D, E) e
clareza mais alta (IF, VVS1) tendem a ser mais caros para o mesmo peso.
A principal dificuldade foi o excesso de combinações (5 cortes × 7
claridades = 35 painéis), tornando o gráfico denso. A solução foi usar
facet\_grid com tamanho ajustado e transparência nos pontos para manter
a legibilidade.

\section{Resumo Final}\label{resumo-final}

Neste laboratório, foi analisada a relação entre o peso e o preço dos
diamantes do conjunto diamonds. Observou-se uma forte correlação
positiva entre essas variáveis, reforçando que o peso é um dos
principais determinantes do preço. Entretanto, a influência de
características como corte, cor e clareza modifica significativamente
essa relação. Cortes de maior qualidade e maior pureza óptica elevam o
preço, mesmo em diamantes de mesmo peso. As técnicas de visualização com
plotnine permitiram explorar essas nuances de forma clara e eficiente,
utilizando camadas, facetas e ajustes estéticos adequados.

\end{document}
